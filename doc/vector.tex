\section{Vectors and Vector-maps}

An {\tt ADCL\_Vector} combined with the {\tt ADCL\_Vmap} specifies the data
structures  to be used during the communication and the actual data. 


ADCL distinguishes between allocating a
vector and registering a vector. The difference is, that in the first case,
the library allocates the memory for the object as specified by the user,
while in the second case the memory has to be allocated by the application.


\begin{verbatim}
int ADCL_Vmap_halo_allocate ( int hwidth, ADCL_Vmap *vec );
int ADCL_Vmap_list_allocate ( int size, int* rcnts, int* displ, ADCL_Vmap *vec );
int ADCL_Vmap_allreduce_allocate ( MPI_Op op, ADCL_Vmap *vec );
int ADCL_Vmap_reduce_allocate ( MPI_Op op, ADCL_Vmap *vec );
int ADCL_Vmap_alltoall_allocate ( int scnt, int rcnt, ADCL_Vmap *vec );
int ADCL_Vmap_all_allocate ( ADCL_Vmap *vec );
int ADCL_Vmap_inplace_allocate ( ADCL_Vmap *vec );
int ADCL_Vmap_free  ( ADCL_Vmap *vec );
\end{verbatim}

\begin{verbatim}
int ADCL_Vector_allocate_generic ( int ndims, int *dims, int nc, ADCL_Vmap vmap,
                           MPI_Datatype dat, void *data, ADCL_Vector *vec )

\end{verbatim}
with
\begin{itemize}
\item {\tt ndims}(IN): number of dimension of the data structure excluding the
  {\tt nc} argument detailed below.
\item {\tt dims}(IN): array of dimension {\tt ndims} containing the extent of
  each dimension including the halo-cells.
\item {\tt nc}(IN): number of elements per entry. Since many scientific codes
  have matrices where each entry of the matrix is a submatrix itself, this
  argument gives the possibility to specify the dimension of the submatrix.
\item {\tt vmap}(IN): .
\item {\tt dat}(IN): basic MPI datatype describing the data type of the matrix
\item {\tt data}(OUT): pointer to the buffer allocated. This pointer will be
  required for calculations within the user code, since the buffer has been
  allocated by ADCL. ADCL guarantees, that a contiguous memory location has
  been allocated for multi-dimensional arrays. Please note, that the buffer
  pointer is {\bf not} the beginning of the data array, but the pointer to the
  multi-dimensional matrix itself. As an example if {\tt ndims}=2 and the
  temporary variable used within ADCL to allocate the 2-D array is called {\tt
    tmp\_matrix}, then the buffer pointer returned in this argument is {\tt
    data = tmp\_matrix}, which is not equal to {\tt data =
    \&(tmp\_matrix[0][0])} in C!
\item {\tt vec}(OUT): handle to ADCL vector object.
\end{itemize}
There is no Fortran interface defined for this routine. For a discussion, why
no Fortran interface is provided for this routine, please refer to the
discussion of {\tt MPI\_Alloc\_mem} in the MPI-2~\cite{mpi2} specification
section 4.1.1.

\hspace{1cm}
\begin{verbatim}
int ADCL_Vector_register_generic ( int ndims, int *dims, int nc, 
                                   ADCL_Vmap vmap, MPI_Datatype dat, 
                                   void *data, ADCL_Vector *vec )

subroutine ADCL_Vector_register_generic ( ndims,  dims, nc, 
                vmap, dat, data, vec, ierror )
integer ndims, nc, vmap, dat, vec, ierror
integer dims(*)
TYPE data(*)
\end{verbatim}
with
\begin{itemize}
\item {\tt ndims}(IN): number of dimension of the data structure excluding the
  {\tt nc} argument detailed below.
\item {\tt dims}(IN): array of dimension {\tt ndims} containing the extent of
  each dimension including the halo-cells.
\item {\tt nc}(IN): number of elements per entry. Since many scientific codes
  have matrices where each entry of the matrix is a submatrix itself, this
  argument gives the possibility to specify the dimension of the submatrix.
\item {\tt vmap}(IN): variable describing which parts of the matrix can be
  used for communication. Currently supported values are {\tt
    ADCL\_VECTOR\_HALO} if a halo-cell structure shall be used, {\tt
    ADCL\_VECTOR\_ALL} if the entire matrix shall be communicated. Not yet
  implemented are support for upper and lower triangular matrices ( {\tt
    ADCL\_VECTOR\_UP\_TRIANGLE} and {\tt ADCL\_VECTOR\_LOW\_TRIANG} ).
\item {\tt dat}(IN): basic MPI datatype describing the data type of the matrix
\item {\tt data}(IN): pointer to the data array. Please note, that the buffer
  pointer has to be the pointer to the multi-dimensional matrix itself, and
  {\bf not} the beginning of the data array. As an example if {\tt ndims}=2
  and the variable used within application is defined as {\tt double
    tmp\_matrix[10][10]}, the buffer pointer passed to ADCL has to be {\tt
    tmp\_matrix}, which is not equal to {\tt \&(tmp\_matrix[0][0])} in C!
\item {\tt vec}(OUT): handle to ADCL vector object.
\end{itemize}

\hspace{1cm}

\begin{verbatim}
int ADCL_Vector_free ( ADCL_Vector *vec );
\end{verbatim}
with
\begin{itemize}
\item {\tt vec}(INOUT): handle to the vector object allocated with {\tt
    ADCL\_Vector\-\_allocate}. It is illegal to call this function with a
  vector object registered with {\tt ADCL\_Vector\_register}. Upon successful
  completion, the handle will be set to {\tt ADCL\_VECTOR\_NULL}.
\end{itemize}
There is no Fortran interface defined for this routine. For a discussion, why
no Fortran interface is provided for this routine, please refer to the
discussion of {\tt MPI\_Alloc\_mem} in the MPI-2~\cite{mpi2} specification
section 4.1.1.  \hspace{1cm}

\begin{verbatim}
int ADCL_Vector_deregister ( ADCL_Vector *vec );

subroutine ADCL_Vector_deregister ( vec, ierror )
integer vec, ierror
\end{verbatim}
with
\begin{itemize}
\item {\tt vec}(INOUT): handle to the vector object registered with {\tt
    ADCL\_Vector\-\_register}. It is illegal to call this function with a
  vector object allocated with {\tt ADCL\_Vector\_allocate}. Upon successful
  completion, the handle will be set to {\tt ADCL\_VECTOR\_NULL}.
\end{itemize}


The second example assumes, that each entry of this matrix contains of a
vector with five elements. The difference between allocating/registering a
two-dimensional matrix using the {\tt nc} argument and allocating/registering
a three-dimensional matrix setting the {\tt nc} argument to zero is, that the
first approach matches a 2-D process topology, while the second approach
matches a 3-D process topology. Thus, setting the {\tt nc} argument to a
non-negative, non-zero value is equivalent to a data decomposition of one
dimension lower than the data object, where the last dimension is not
distributed across the processes. Please note in the example below the change
in the dimension of the {\tt data} matrix.

\begin{verbatim}
#include <stdio.h>
#include "ADCL.h"
#include "mpi.h"

/* Dimensions of the data matrix per process */
#define DIM0  8
#define DIM1  4

int main ( int argc, char ** argv ) 
{
    int dims[2];
    int nc=5, hwidth=1;
    double ***data;
    ADCL_Vector vec;
    
    MPI_Init ( &argc, &argv );
    ADCL_Init ();
    
    dims[0] = DIM0 + 2*hwidth;
    dims[1] = DIM1 + 2*hwidth;
    ADCL_Vector_allocate ( 2,  dims, nc, ADCL_VECTOR_HALO, 
        hwidth, MPI_DOUBLE, &data, &vec );
    
    /* now you can access the elements of the vector in order to perform 
       computations, e.g */
    for ( i=1; i<DIM0; i++ ) {
       for ( j=1; j<DIM1; j++) {
         for ( k=0; k < nc; k++ ) {
           data[i][j][k] = ...
       }
    }
    
    ADCL_Vector_free ( &vec );        
    ADCL_Finalize ();
    MPI_Finalize ();
    return 0;
}
\end{verbatim}


