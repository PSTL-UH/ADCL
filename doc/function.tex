\section{Functions and Function-sets}

An {\tt ADCL\_Function} is the equivalent to an actual implementation of a
particular communication pattern. An ADCL Function can have an attribute-set
attached to it, in which case the values for each of the attributes in the
attribute-set for this particular function have to be defined. A user can
however also decide not to attach an attribute-set to a function by passing in
{\tt ADCL\_ATTRSET\_NULL} at the particular argument.

An {\tt ADCL\_Fnctset} is a collection of ADCL functions providing the same
functionality. All Functions in a function-set have to have the same
attribute-set. ADCL provides pre-defined function sets, see
section~\ref{fnctset-predef} for a list of predefined function-sets. The user
can also register its own functions in order to utilize the ADCL runtime
selection logic.

\begin{verbatim}
typedef void ADCL_work_fnct_ptr ( ADCL_Request req );

int ADCL_Function_create ( ADCL_work_fnct_ptr *fnctp, 
         ADCL_Attrset attrset, int *array_of_attrvalues,  
         char *name, ADCL_Function *fnct);

subroutine ADCL_Function_create ( fnctp, attrset, 
                array_of_attrvalues, name, fnct, ierror )
external fnctp
integer attrset, fnct, ierror
integer array_of_attrvalues(*)
char name (*)	                                

\end{verbatim}
with
\begin{itemize}
\item {\tt fnctp}(IN): function pointer to the actual implementation. The
  prototype has to be of type {\tt ADCL\_work\_fnct\_ptr}.

\item {\tt attrset}(IN): valid ADCL attribute-set handle, or {\tt
  ADCL\_ATTRSET\_NULL}. Passing the NULL attribute-set object in forces ADCL
  to use the brute-force runtime search algorithm.

\item {\tt array\_of\_attrvalues}(IN): if an attribute-set has been specified,
  this array of integers has to provide the values for each attribute in the
  attribute-set.

\item {\tt name}(IN): name for the function. The length of the character
  string can not exceed {\tt ADCL\_MAX\_NAMELEN}. It is allowed to pass in a
  NULL pointer instead of a string.

\item {\tt fnct}(OUT): handle to the ADCL function object.
\end{itemize}

\hspace{1cm}
\begin{verbatim}
int ADCL_Function_create_async ( ADCL_work_fnct_ptr *init_fnct, 
         ADCL_work_fnct_ptr *wait_fnct, 
         ADCL_Attrset attrset, int *array_of_attrvalues, 
         char *name, ADCL_Function *fnct);

subroutine ADCL_Function_create_async ( init_fnct, wait_fnct, 
                attrset, array_of_attrvalues, name, fnct, ierror )
external init_fnct, wait_fnct
integer attrset, fnct, ierror
integer array_of_attrvalues(*)
char name (*)	  
\end{verbatim}
with
\begin{itemize}

\item {\tt init\_fnct}(IN): function pointer to the actual implementation of
  the initiation function. The prototype has to be of type {\tt
    ADCL\_work\_fnct\_ptr}.

\item {\tt wait\_fnct}(IN): function pointer to the actual implementation of
  the completion function. The prototype has to be of type {\tt
    ADCL\_work\_fnct\_ptr}.

\item {\tt attrset}(IN): valid ADCL attribute-set handle, or {\tt
  ADCL\_ATTRSET\_NULL}. Passing the NULL attribute-set object in forces ADCL
  to use the brute-force runtime search algorithm.

\item {\tt array\_of\_attrvalues}(IN): if an attribute-set has been specified,
  this array of integers has to provide the values for each attribute in the
  attribute-set.

\item {\tt name}(IN): name for the function. The length of the character
  string can not exceed {\tt ADCL\_MAX\_NAMELEN}. It is allowed to pass in a
  NULL pointer instead of a string.

\item {\tt fnct}(OUT): handle to the ADCL function object.
\end{itemize}


\hspace{1cm}
\begin{verbatim}
int ADCL_Function_free ( ADCL_Function *fnct );

subroutine ADCL_Function_free ( fnct, ierror )
integer fnct, ierror
\end{verbatim}
with
\begin{itemize}

\item {\tt fnct}(INOUT): valid handle to an ADCL function. Upon return, the
  handle is set to {\tt ADCL\_FUNCTION\_NULL}
\end{itemize}

\hspace{1cm}
\begin{verbatim}
int ADCL_Fnctset_create ( int maxnum, ADCL_Function *fncts, 
         char *name, ADCL_Fnctset *fnctset );

subroutine ADCL_Fnctset_create ( maxnum, fncts, name, 
                fnctset, ierror )
integer maxnum, fnctset, ierror
integer fncts(*)
char name (*)
\end{verbatim}
with
\begin{itemize}
\item {\tt maxnum}(IN): number of ADCL functions to be bundled to a function set
\item {\tt fncts}(IN): array of size {\tt maxnum} containing the handles to
  the ADCL functions. All functions have to provide the same attribute-set.
\item {\tt name}(IN): name of the function set. It is allowed to pass in a
  NULL pointer instead of a name. The length of the character string can not
  exceed {\tt ADCL\_MAX\_NAMELEN}.
\item {\tt fnctset}(OUT): handle for the ADCL function-set.
\end{itemize}


The following function provides a short cut for scenarios, where a single function
shall be executed with different attribute values. Thus, instead of the user
having to create all the individual funtions and register them with ADCL, this
interface provides the opportunity to specify the function pointer, an
attribute set, and get an Functionset handle back. ADCL will create internally
a function with the same function pointer for each possible combination of
attribute values. Some combinations of attribute values can be also excluded.

\begin{verbatim}
int ADCL_Fnctset_create_single ( ADCL_work_fnct_ptr *init_fnct,
         ADCL_work_fnct_ptr *wait_fnct, ADCL_Attrset attrset, 
         char *name, int **without_attr_combinations, 
         int num_without_attr_combinations, 
         ADCL_Fnctset *fnctset );

subroutine ADCL_Fnctset_create_single ( void *init_fnct, 
                void *wait_fnct, int *attrset, char *name,
                int *without_attr_vals, 
                int *num_without_attr_vals,
                int *fnctset, int *ierror, int name_len );
external init_fnct, wait_fnct
integer attrset, without_attr_vals, num_without_attr_vals
integer fnctset, ierror
char name (*)
\end{verbatim}
with
\begin{itemize}

\item {\tt init\_fnct}(IN): function pointer to the actual implementation of
  the init function. The prototype has to be of type {\tt
    ADCL\_work\_fnct\_ptr}.

\item {\tt wait\_fnct}(IN): function pointer to the actual implementation of
  the completion function. The prototype has to be of type {\tt
    ADCL\_work\_fnct\_ptr}.  It is allowed to pass in a NULL pointer if there
  is no need to a completion function.

\item {\tt attrset}(IN): valid ADCL attribute-set handle, or {\tt
  ADCL\_ATTRSET\_NULL}. It is illegal for this function to pass in the NULL
  attribute-set handle.

\item {\tt name}(IN): name for the function. The length of the character
  string can not exceed {\tt ADCL\_MAX\_NAMELEN}. It is allowed to pass in a
  NULL pointer instead of a string.

\item {\tt without\_attr\_vals}(IN): array containing the attribute values
  that the user want to exclude.

\item {\tt num\_without\_attr\_vals}(IN): number of entries in the array
  containing the attribute values that the user want to exclude.

\item {\tt fnctset}(OUT): handle to the ADCL function-set object.
\end{itemize}
\hspace{1cm}
\begin{verbatim}
int ADCL_Fnctset_free   ( ADCL_Fnctset *fnctset );

subroutine ADCL_Fnctset_free ( fnctset, ierror )
integer fnctset, ierror
\end{verbatim}
with
\begin{itemize}

\item {\tt fnctset}(INOUT): valid ADCL function-set handle to be freed. After
  successful completion, the handle is set to {\tt ADCL\_FNCTSET\_NULL}.
\end{itemize}


\pagebreak
\subsection{Predefined Function-sets}
\label{fnctset-predef}

ADCL currently supports the following  predefined function-sets: 
\begin{itemize}
\item {\tt ADCL\_FNCTSET\_NEIGHBORHOOD}: a function-set supporting
  n-dimensional neighborhood communication. The dimension of the neighborhood
  communication is determined by the dimension of the vector object used
  during the request creation time.
\item {\tt ADCL\_FNCTSET\_ALLGATHERV}: a function-set supporting an
  MPI\_Allgatherv style communication.
\item {\tt ADCL\_FNCTSET\_ALLREDUCE}: a function-set supporting an
  MPI\_Allreduce style communication.
\item {\tt ADCL\_FNCTSET\_REDUCE}: a function-set supporting an
  MPI\_Reduce style communication.
\item {\tt ADCL\_FNCTSET\_ALLTOALL}: a function-set supporting an
  MPI\_Alltoall style communication.
\item {\tt ADCL\_FNCTSET\_ALLTOALLVV}: a function-set supporting an
  MPI\_Alltoallv style communication.
\item {\tt ADCL\_FNCTSET\_IBCAST}: a function-set supporting a non-blocking
  broadcast operation similarly to MPI\_Ibcast. Requires a functional
  implementation of the LibNBC library.
\item {\tt ADCL\_FNCTSET\_IALLTOALL}: a function-set supporting a non-blocking
  all-to-all operation similarly to MPI\_IAlltoall. Requires a functional
  implementation of the LibNBC library.
 
\end{itemize}

Other predefined function-sets might be added in following releases of ADCL.

\subsection{Examples}

The following is an example for a code registering three functions without an
attribute-set, and combining them to a function-set. Since there is no
attribute-set attached to the functions, the array of attribute-values
argument of the ADCL function constructors can be a NULL pointer. Please note,
that not setting an attribute-set forces ADCL to fall back to the brute force
runtime selection logic.
\begin{verbatim}
#include <stdio.h>
#include "ADCL.h"
#include "mpi.h"

void test_func_1 ( ADCL_Request req );
void test_func_2 ( ADCL_Request req );
void test_func_3 ( ADCL_Request req );

int main ( int argc, char ** argv ) 
{
    ADCL_Function funcs[3];
    ADCL_Fnctset fnctset;
    int i;
        
    MPI_Init ( &argc, &argv );
    ADCL_Init ();

    ADCL_Function_create ( (ADCL_work_fnct_ptr *)test_func_1, 
         ADCL_ATTRSET_NULL, NULL, "test_func_1", &(funcs[0]));
    ADCL_Function_create ( (ADCL_work_fnct_ptr *)test_func_2, 
         ADCL_ATTRSET_NULL, NULL, "test_func_2", &(funcs[1]));
    ADCL_Function_create ( (ADCL_work_fnct_ptr *)test_func_3, 
         ADCL_ATTRSET_NULL, NULL, "test_func_3", &(funcs[2]));

    ADCL_Fnctset_create ( 3, funcs, "trivial functions", 
         &fnctset );

    /* Do something with the fnctset */
    
    ADCL_Fnctset_free ( &fnctset );
    for ( i=0; i<3; i++ ) {
      	ADCL_Function_free ( &funcs[i] );
    }    
    ADCL_Finalize ();
    MPI_Finalize ();
    return 0;
}

void test_func_1 ( ADCL_Request req ) {
    int rank;
    
    MPI_Comm_rank ( MPI_COMM_WORLD, &rank );
    printf("%d: In test_func_1 \n", rank);
    return;
}

void test_func_2 ( ADCL_Request req ) {
    int rank;
    
    MPI_Comm_rank ( MPI_COMM_WORLD, &rank );
    printf("%d: In test_func_2\n", rank);
    return;
}

void test_func_3 ( ADCL_Request req ) {
    int rank;
    
    MPI_Comm_rank ( MPI_COMM_WORLD, &rank ); 
    printf("%d: In test_func_3\n", rank);
    return;
}

\end{verbatim}

