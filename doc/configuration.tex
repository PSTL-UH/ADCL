\section{Runtime Configuration}

There are multiple ways on how to influence or enforce a particular behavior
of ADCL at runtime. ADCL reads a configuration file {\tt config.adcl}, from
the same director as the executable is located (1st option) or from {\tt
  \$(HOME)/.adcl} (2nd option).

The configuration file has the format:
\begin{verbatim}
OPTION: value
\end{verbatim}

with the OPTION being a key word recognized by ADCL, and value being a valid
value for that key word. Currently recognized key words and there
corresponding valid values are as follows:

\paragraph{Generic parameters influencing the behavior of ADCL}
\begin{itemize}
\item {\tt ADCL\_MERGE\_REQUESTS}: value indicating whether multiple requests
that have identical functionsets and vmap objects, but differ in the vector
object (i.e. the actual array used is different but everything else is
identical) shall be co-tuned or tuned separately. Valid values: 1 (enable
co-tuning) or 0 (disable co-tuning).

\item{\tt ADCL\_PRINTF\_SILENCE}: whether ADCL should log the the tuning into into
  the per process output files (rank.out) or not. Valid values: 0 (write
  logging), 1 (dont write logging).

\item {\tt ADCL\_EMETHOD\_SEARCH\_ALGO}: search algorithm to be used for
  tuning. Valid values: 0 (brute force search), 1 (orthogonal attribute based
  search), 2 ( 2k factorial design based search). Note, that only function
  sets with attributes can use algorithms 1 and 2.

\item{\tt ADCL\_EMETHOD\_NUMTESTS}: number of measurements to be performed per
  function. Valid values: anything between 1 and the value provided at
  configure time (default: 30).

\item{\tt ADCL\_USE\_BARRIER}: Valid values: 0 ( don't use  MPI\_Barrier
  before every invocation of a function), 1 (use a barrier before every
  invocation of a function).

\item {\tt ADCL\_TIMER\_STEPS\_BETWEEN\_BARRIER}: t.b.d.

\item{\tt ADCL\_OUTLIER\_FRACTION}: percentage of measurements that are
  allowed to be classified as outliers. Valid values: 
\item{\tt ADCL\_OUTLIER\_FACTOR}:  An outlier is defined as a data point that
  is {\tt ADCL\_OUTLIER\_FACTOR} times larger than the minimum value on a per
  process basis.
%\item{\tt ADCL\_STATISTIC\_METHOD}: algorithm used to determine
%  outliers. Valid values are: ADCL\_STATISTIC\_MAX. REmoved from document
% for now since it only has one option.

\end{itemize}


\paragraph{Parameters influencing a particular function set}
\begin{itemize}
\item{\tt ADCL\_EMETHOD\_ALLGATHERV\_SELECTION}: force the usage of a particular
  implementation for the Allgatherv function set. Setting this value
  disables runtime tuning of this function set. Valid values: Allgatherv\_bruck, Allgatherv\_linear,
  Allgatherv\_native, Allgatherv\_neighbor\_exchange,
  Allgatherv\_recursive\_doubling, Allgatherv\_ring

\item{\tt ADCL\_EMETHOD\_ALLREDUCE\_SELECTION}: force the usage of a
  particular implementation for the Allreduce function set. Setting this value
  disables runtime tuning of this function set. Valid values:
  Allreduce\_native, Allreduce\_linear, Allreduce\_ring,
  Allreduce\_recoursive\_doubling, Allreduce\_nonoverlapping.

\item{\tt ADCL\_EMETHOD\_REDUCE\_SELECTION}: t.b.d.

\item{\tt ADCL\_EMETHOD\_ALLTOALL\_SELECTION}:force the usage of a
  particular implementation for the Alltoall function set. Setting this value
  disables runtime tuning of this function set. Valid values:
  Alltoall\_native\_SR, Alltoall\_bruck\_SR, Alltoall\_ladd\_block2,
  Alltoall\_ladd\_block4, Alltoall\_ladd\_block8, Alltoall\_linear\_SR,
  Alltoall\_linear\_sync\_SR, Alltoall\_pair\_excl\_SR.

\item{\tt ADCL\_EMETHOD\_ALLTOALLV\_SELECTION}: t.b.d.

\item {\tt ADCL\_EMETHOD\_SELECTION}:force the usage of a particular
  implementation for the Neighborhood function set. Setting this value
  disables runtime tuning of this function set. Valid values: IsendIrecv\_aao,
  IsendIrecv\_aao\_pack, IsendIrecv\_pair, IsendIrecv\_pair\_pack,
  SendIrecv\_aao, SendIrecv\_aao\_pack, SendIrecv\_pair, SendIrecv\_pair\_pack,
  Sendrecv\_pair, Sendrecv\_pair\_pack, SendRecv\_pair,
  SendRecv\_pair\_pack. 

  If one sided communication has been enabled at configure time, the following
  values can also be used for neighborhood commmunication: PostStartGet\_aao,
  PostStartGet\_aao\_pair, PostStartPut\_aao, PostStartPut\_pair,
  WinFenceGet\_pair, WinFenceGet\_aao, WinFencePut\_pair, WinFencePut\_aao.
 
\item{\tt ADCL\_EMETHOD\_IBCAST\_SELECTION}: to be done by Youcef.
\item{\tt ADCL\_EMETHOD\_IALLTOALL\_SELECTION}: to be done by Youcef.
\end{itemize}


\paragraph{Parameters influencing the ADCL GUI}
\begin{itemize}
\item {\tt ADCL\_DISPLAY\_IP}: IP address of the host running the ADCL GUI.
\item{\tt ADCL\_DISPLAY\_PORT}: TCP port used by the ADCL GUI.
\item{\tt ADCL\_DISPLAY\_RANK}: comma separated list of MPI ranks that should connect to the
  ADCL GUI. 
\end{itemize}
  

\paragraph{Parameters influencing Historic Learning}
\begin{itemize}
\item{\tt ADCL\_HIST\_LEARNING}: controls the utilization of historic
  learning. Valid values: 0 ( disable historic learning), 1 (enable historic
  learning).

\item{\tt ADCL\_HIST\_PREDICTOR}: prediction algorithm used if historic
  learning has been enabled. Valid values: 0 (no clue), 1 (need to look it
  up).
\item{\tt ADCL\_HIST\_CLUSTER}:
\end{itemize}
