\section{Installation}

ADCL follows the regular UNIX/LINUX software configuration and installation
model, requiring a sequence of 

\begin{verbatim}
configure
make
make install
\end{verbatim}

\subsection{Prerequisites}

ADCL requires a working installation of an MPI
library, complying to the MPI-2 standard. ADCL has been tested with a large
number of MPI libraries and on multiple platforms, including Open MPI, MPICH,
MVAPICH, IBM MPI, and Cray MPI.

For the non-blocking operations supported by ADCL, a valid and working
installation of the libNBC library has to be provided, compiled with the same
MPI library as the one being used for ADCL.

The GNU scientific library is used for some clusternig algorithm, and can
optionally be provided. The PAPI hardware performance counter library can be
utilized by ADCL for high resolution timers.

\subsection{Configure Options}
ADCL can be customized with a number of options, which can be devided into
three sections: enabling or disabling certain features, providing additional
libraries (e.g. libNBC, GSL, PAPI) to enabel certain features, and change the
default behavior of ADCL. The following is a list of all configure options
supported by ADCL.

\small
\begin{verbatim}
Configuration:
  -h, --help              display this help and exit
      --help=short        display options specific to this package
      --help=recursive    display the short help of all the included packages
  -V, --version           display version information and exit
  -q, --quiet, --silent   do not print `checking ...' messages
      --cache-file=FILE   cache test results in FILE [disabled]
  -C, --config-cache      alias for `--cache-file=config.cache'
  -n, --no-create         do not create output files
      --srcdir=DIR        find the sources in DIR [configure dir or `..']

Installation directories:
  --prefix=PREFIX         install architecture-independent files in PREFIX
                          [/home/gabriel/ADCL]
  --exec-prefix=EPREFIX   install architecture-dependent files in EPREFIX
                          [PREFIX]

By default, `make install' will install all the files in
`/home/gabriel/ADCL/bin', `/home/gabriel/ADCL/lib' etc.  You can specify
an installation prefix other than `/home/gabriel/ADCL' using `--prefix',
for instance `--prefix=$HOME'.

For better control, use the options below.

Fine tuning of the installation directories:
  --bindir=DIR            user executables [EPREFIX/bin]
  --sbindir=DIR           system admin executables [EPREFIX/sbin]
  --libexecdir=DIR        program executables [EPREFIX/libexec]
  --sysconfdir=DIR        read-only single-machine data [PREFIX/etc]
  --sharedstatedir=DIR    modifiable architecture-independent data [PREFIX/com]
  --localstatedir=DIR     modifiable single-machine data [PREFIX/var]
  --libdir=DIR            object code libraries [EPREFIX/lib]
  --includedir=DIR        C header files [PREFIX/include]
  --oldincludedir=DIR     C header files for non-gcc [/usr/include]
  --datarootdir=DIR       read-only arch.-independent data root [PREFIX/share]
  --datadir=DIR           read-only architecture-independent data [DATAROOTDIR]
  --infodir=DIR           info documentation [DATAROOTDIR/info]
  --localedir=DIR         locale-dependent data [DATAROOTDIR/locale]
  --mandir=DIR            man documentation [DATAROOTDIR/man]
  --docdir=DIR            documentation root [DATAROOTDIR/doc/PACKAGE]
  --htmldir=DIR           html documentation [DOCDIR]
  --dvidir=DIR            dvi documentation [DOCDIR]
  --pdfdir=DIR            pdf documentation [DOCDIR]
  --psdir=DIR             ps documentation [DOCDIR]

System types:
  --build=BUILD     configure for building on BUILD [guessed]
  --host=HOST       cross-compile to build programs to run on HOST [BUILD]
  --target=TARGET   configure for building compilers for TARGET [HOST]

Optional Features:
  --disable-option-checking  ignore unrecognized --enable/--with options
  --disable-FEATURE       do not include FEATURE (same as --enable-FEATURE=no)
  --enable-FEATURE[=ARG]  include FEATURE [ARG=yes]
  --enable-gsl                      enable the usage of gsl (default=no)
  --enable-libnbc                   enable the usage of libNBC (default=no)
  --enable-onesided                 enable the usage of one-sided communication (default=no)
  --disable-printf-tofile           dump printf statements to a file (default=yes)
  --enable-userlevel-timings        disable ADCL internal timing routines (default=no)
  --enable-knowledge-tofile         dump the ADCL knowledge to an XML file (default=no)
  --disable-saving-request-winner   save the winner of a request (default=yes)
  --enable-smooth-hist              smooth the history data (default=no)
  --enable-dummy-mpi                enable the usage of dummy-mpi (default=no)
  --enable-cluster-classify         enable the usage of clustering and
                                    classification with SVM for historic learning (default=no)
  --disable-fortran                 compile fortran wrappers (default=yes)

Optional Packages:
  --with-PACKAGE[=ARG]              use PACKAGE [ARG=yes]
  --without-PACKAGE                 do not use PACKAGE (same as --with-PACKAGE=no)
  --with-adcl-dir=dir               Main ADCL directory (default=$PWD)
  --with-adcl-inc-dir=dir           ADCL include directory (default=ADCL_DIR/include)
  --with-adcl-lib-dir=dir           ADCL lib directory (default=ADCL_DIR/lib)
  --with-adcl-bin-dir=dir           ADCL bin directory (default=ADCL_DIR/bin)
  --with-adcl-lib-name=name         Name of the ADCL library (default=libadcl.a)
  --with-adcl-timer=timer           Timer options (default=TIMER_GETTIMEOFDAY)
  --with-mpi-dir=dir                Installation directory of MPI
  --with-mpi-inc-dir=dir            Include directory of MPI (default=MPI_DIR/include)
  --with-mpi-lib-dir=dir            Library directory of MPI (default=MPI_DIR/lib)
  --with-mpi-bin-dir=dir            Bin directory of MPI (default=MPI_DIR/bin)
  --with-mpi-cc=mpicc               MPI C compiler wrapper (default=mpicc)
  --with-mpi-cxx=mpicxx             MPI C++ compiler wrapper (default=mpiCC)
  --with-mpi-f90=mpif90             MPI f90 compiler wrapper (default=mpif90)
  --with-papi-dir=dir               Main PAPI directory (default=$PWD)
  --with-papi-inc-dir=dir           PAPI include directory (default=PAPI_DIR/include)
  --with-papi-lib-dir=dir           PAPI lib directory (default=PAPI_DIR/lib)
  --with-papi-lib-name=name         Name of the PAPI library (default=papi)
  --with-gsl-dir=dir                Main GSL directory (default=/usr/local)
  --with-gsl-inc-dir=dir            GSL include directory (default=GSL_DIR/include/gsl)
  --with-gsl-lib-dir=dir            GSL lib directory (default=GSL_DIR/lib)
  --with-libnbc-dir=dir             Main libNBC directory (default=/usr/local)
  --with-libnbc-inc-dir=dir         libNBC include directory (default=LIBNBC_DIR/include)
  --with-libnbc-lib-dir=dir         libNBC lib directory (default=LIBNBC_DIR/lib)
  --with-num-tests=NUMTESTS         Number of measurements per implementation (default=30)
  --with-perf-win=PERF-WIN          Acceptable performance window (default=10)
  --with-pred-algo=algorithm        Prediction algorithm options (default=ADCL_WMV)
  --with-smooth-win=WIN-SIZE        Size of the smoothing window (default=3)

Some influential environment variables:
  CC          C compiler command
  CFLAGS      C compiler flags
  LDFLAGS     linker flags, e.g. -L<lib dir> if you have libraries in a
              nonstandard directory <lib dir>
  LIBS        libraries to pass to the linker, e.g. -l<library>
  CPPFLAGS    (Objective) C/C++ preprocessor flags, e.g. -I<include dir> if
              you have headers in a nonstandard directory <include dir>
  CXX         C++ compiler command
  CXXFLAGS    C++ compiler flags
  F77         Fortran 77 compiler command
  FFLAGS      Fortran 77 compiler flags

Use these variables to override the choices made by `configure' or to help
it to find libraries and programs with nonstandard names/locations.

Report bugs to the package provider.
\end{verbatim}
