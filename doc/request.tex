\section{Requests}
\label{sec:requests}

An {\tt ADCL\_Request} object combines a process topology, a function-set and
a vector object to a single user-level handle. Using this handle, the
application can initiate a communication by starting a particular ADCL
request. In the following we will detail request constructors and destructors,
the available functions for initiating a communication, and reflection
functions in order to retrieve certain components of an ADCL request from a
user level function.

\subsection{Constructors and Destructors}

\begin{verbatim}
int ADCL_Request_create ( ADCL_Vector vec, ADCL_Topology topo, 
         ADCL_Fnctset fnctset, ADCL_Request *req );

subroutine ADCL_Request_create ( vec, topo, fnctset, req, ierror)
integer vec, topo, fnctset, req, ierror
\end{verbatim}
with
\begin{itemize}
\item {\tt vec}(IN): vector object identifying the data arrays used for the
  communication. The vector object can be {\tt ADCL\_VECTOR\_NULL} for a user
  defined function-set following the requirements and restrictions of the
  according function-set. However, for the predefined function-sets, the
  vector object has to be provided following the dimension matching rule
  explained below.
\item {\tt topo}(IN): topology object identifying the neighboring processes.
  The topology object can be {\tt ADCL\_TOPOLOGY\_NULL} for user defined
  function-set following the requirements and restrictions of the according
  function-set. However, for the predefined function-sets, the topology object
  has to be provided following the dimension matching rule explained below.
\item {\tt fnctset}(IN): ADCL function-set to be evaluated
\item {\tt req}(OUT): handle to the newly created ADCL request object
\end{itemize}
\hspace{1cm}

{\it Dimension matching rule}: The dimension of the vector object has to be
equal to the dimension of the process topology. The user has one additional
degree of freedom by using the {\tt nc} argument in vector creation, which
allows to match a vector of dimension $n+1$ to a $n$ dimensional process
topology, by not distributing the last dimension of the vector.

{\it Remark on the dimension matching rule}: It is fairly straight forward to
define a more generic vector object, where the user could define a
multi-dimensional vector and assign to each of the dimensions a flag which
indicates whether the data is distributed accordingly in this dimension or not.
The solution applied so far restricts this approach by reducing the dimension
of a matrix which is not distributed across the processes to the last
dimension. This is due to our findings, that it is a common approach in
many codes. However, if an end-user faces a problem requiring the generic
solution, please contact the authors of ADCL. In any case, the number of
dimension of the matrix distributed across the process has to match the number
of dimensions of the process topology.

\begin{verbatim}
int ADCL_Request_create_generic ( ADCL_Vectset vecset, 
         ADCL_Topology topo, ADCL_Fnctset fnctset, 
         ADCL_Request *req );
      
subroutine ADCL_Request_create_generic ( vecset, topo, 
                fnctset, req, ierror )
integer vecset, topo, fnctset, req, ierror
\end{verbatim}
with
\begin{itemize}
\item {\tt vecset}(IN): vector set to be used in the request. The extent of
  the array of send vectors and receive vectors has to match the number of
  dimensions of the topology object.
\item {\tt topo}(IN): topology object identifying the neighboring processes.
  The topology object can be {\tt ADCL\_TOPOLOGY\_NULL} for user defined
  function-set following the requirements and restrictions of the according
  function-set. However, for the predefined function-sets, the topology object
  has to be provided following the dimension matching rule explained below.
\item {\tt fnctset}(IN): ADCL function-set to be evaluated
\item {\tt req}(OUT): handle to the newly created ADCL request object
\end{itemize}
\hspace{1cm}

\begin{verbatim}
int ADCL_Request_free ( ADCL_Request *req );

subroutine ADCL_Request_free ( req, ierror )
integer req, ierror
\end{verbatim}
with
\begin{itemize}
\item {\tt req}(INOUT): valid ADCL request handle. After successful
  completion, the handle will be set to {\tt ADCL\_REQUEST\_NULL}.
\end{itemize}
\hspace{1cm}

\subsection{Initiating communication operations}

The following function can be used to start a blocking instance of the communication pattern.
\begin{verbatim}
int ADCL_Request_start ( ADCL_Request req );

subroutine ADCL_Request_start ( req, ierror )
integer req, ierror
\end{verbatim}
with
\begin{itemize}
\item {\tt req}(IN): valid ADCL request
\end{itemize}
\hspace{1cm}

The following two functions represent a non-blocking version of {\tt
  ADCL\_Request\_start}.  There are no restrictions in ADCL on how many ADCL
requests can be ongoing simultaneously. However, for a single request a call
to {\tt ADCL\_Request\_init} has to be followed by a call to {\tt
  ADCL\_Request\_wait}.
\begin{verbatim}
int ADCL_Request_init  ( ADCL_Request req );

subroutine ADCL_Request_init ( req, ierror )
integer req, ierror
\end{verbatim}
with
\begin{itemize}
\item {\tt req}(IN): valid ADCL request
\end{itemize}
\hspace{1cm}

\begin{verbatim}
int ADCL_Request_wait  ( ADCL_Request req );

subroutine ADCL_Request_wait ( req, ierror )
integer req, ierror
\end{verbatim}
with
\begin{itemize}
\item {\tt req}(IN): valid ADCL request started with {\tt ADCL\_Request\_init}.
\end{itemize}
\hspace{1cm}

%The following is an interface, where the user pass in additional function
%pointers in order to evaluate the effect of overlapping communication and
%computation. The proper interface is still being investigated and might
%change.
%
%\begin{verbatim}
%int ADCL_Request_start_overlap ( ADCL_Request req, 
%         ADCL_work_fnct_ptr* midfctn,
%         ADCL_work_fnct_ptr *endfcnt, 
%	 ADCL_work_fnct_ptr *totalfcnt );
%
%subroutine ADCL_Request_start_overlap ( req, midfctn,
%                endfnct, totalfnct, ierror )
%integer req, ierror
%external midfnct, endfnct, totalfnct
%\end{verbatim}
%with
%\begin{itemize}
%\item {\tt req}(IN): valid ADCL request handle
%\item {\tt midfnct}(IN): function to be called after initiating communication
%\item {\tt endfnct}(IN): function to be called after communication is finished
%\item {\tt totalfnct}(IN): function to be called in case the communication is
%  blocking. Typically, {\tt totalfnct} will be an integrated version of {\tt
%    midfnct} plus {\tt endfnct}.
%\end{itemize}
%\hspace{1cm}

\subsection{Retrieving internal components}

The following function helps to retrieve the communicator object assigned to a
request. It is intended to be used by user registered ADCL functions
internally. Similar functions for retrieving data pointers will be introduced
soon

\begin{verbatim}
int ADCL_Request_get_comm  ( ADCL_Request req, 
         MPI_Comm *comm, int *rank, int *size );


subroutine ADCL_Request_get_comm ( req, comm, rank,
                size, ierror )
integer req, comm, rank, size, ierror     
\end{verbatim}
with
\begin{itemize}
\item {\tt req}(IN): valid ADCL request handle
\item {\tt comm}(OUT): communicator used when creating the according ADCL topology object
\item {\tt rank}(OUT): rank of the process in the according communicator {\tt comm}.
\item {\tt size}(OUT): size of the communicator {\tt comm}.
\end{itemize}
\hspace{1cm}

\begin{verbatim}
int ADCL_Request_get_curr_function ( ADCL_Request req, 
         char **function_name, char ***attribute_names, 
         int *num_of_attributes, char ***attribute_values_names, 
         int **attribute_values );
\end{verbatim}
with
\begin{itemize}
\item {\tt req}(IN): valid ADCL request handle
\item {\tt function\_name}(OUT): Name of the function currently being
  called. This string has to be freed by the user level function.
\item {\tt attribute\_names}(OUT):
\item {\tt num\_of\_attributes}(OUT): number of attributes. This value is equal
  to the extent of the {\tt attribute\_names}, {\tt attribute\_values\_names}
  and {\tt attribute\_values} arrays.
\item {\tt attribute\_values\_names}(OUT): array of names for the attribute values
  of the current function.
\item {\tt attribute\_values}(OUT): array of integers containing the attribute
  values of the current function.
\end{itemize}
\hspace{1cm}

%\begin{verbatim} 
%int ADCL_Request_get_winner_stat ( ADCL_Request req, 
%         double *filtered_avg, double *unfiltered_avg,
%         double *outliers_num );
%\end{verbatim}
%with
%\begin{itemize}
%\item {\tt req}(IN): valid ADCL request handle
%\item {\tt filtered\_avg}(OUT): Filtered average execution time of the winning ADCL function.
%\item {\tt unfiltered\_avg}(OUT): unfiltered average execution time of the winning ADCL function.
%\item {\tt outliers\_num}(OUT): number of outliers detected during the execution the winning ADCL %function.
%\end{itemize}

%\begin{verbatim}
%int ADCL_Request_get_vec ( ADCL_Request req, 
%    int *ndims, void *dims, int *nc, int *hwidth,
%    MPI_Datatype *dat, void *data );
%    
%subroutine ADCL_Request_get_vec ( req, ndims, 
%   dims, nc, hwidth, dat, data )
%integer req, ndims, nc, hwidth, dat
%integer dims(*)
%TYPE data(*)    
%\end{verbatim}
%with
%\begin{itemize}
%\item {\tt req}(IN):
%\item {\tt ndims}(OUT):
%\item {\tt dims}(OUT):
%\item {\tt nc}(OUT):
%\item {\tt hwidth}(OUT):
%\item {\tt dat}(OUT):
%\item {\tt data}(OUT):
%\end{itemize}
%\hspace{1cm}

\subsection{Save/Restore request status}
The following function helps the user to save the progress in the search process of the fastest implementation when using the brute force strategy. Then, the user can simply resume the search from where he stopped by restoring the request status using the function {\tt ADCL\_Request\_restore\_status}.

\begin{verbatim}
int ADCL_Request_save_status ( ADCL_Request req, int *tested_num,
         double **unfiltered_avg, double **filtered_avg,
         double **outliers, int *winner_so_far );
\end{verbatim}
with
\begin{itemize}
\item {\tt req}(IN): valid ADCL request handle
\item {\tt tested\_num}(OUT): number of already evaluated ADCL functions.
\item {\tt unfiltered\_avg}(OUT): unfiltered average of the evaluated ADCL functions.
\item {\tt filtered\_avg}(OUT): filtered average of the evaluated ADCL functions.
\item {\tt outliers}(OUT): number of outliers detected in the evaluated ADCL functions.
\item {\tt winner\_so\_far}(OUT): the winning implementation so far.
\end{itemize}
\begin{verbatim}
int ADCL_Request_restore_status ( ADCL_Request req, int tested_num,
         double *unfiltered_avg, double *filtered_avg,
         double *outliers );
\end{verbatim} 
with
\begin{itemize}
\item {\tt req}(IN): valid ADCL request handle
\item {\tt tested\_num}(IN): number of already evaluated ADCL functions.
\item {\tt unfiltered\_avg}(IN): unfiltered average of each evaluated ADCL functions.
\item {\tt filtered\_avg}(IN): filtered average of each evaluated ADCL functions.
\item {\tt outliers}(IN): number of outliers detected in the evaluated ADCL functions.
\end{itemize}
\subsection{Examples}

The following example demonstrates the usage of the predefined function set {\tt ADCL\_FNCTSET\_NEIGHBORHOOD} for a three dimensional matrix and process topology.

\begin{verbatim}
#include <stdio.h>
#include "ADCL.h"
#include "mpi.h"

/* Dimensions of the data matrix per process */
#define DIM0  64
#define DIM1  32
#define DIM2  32

int main ( int argc, char ** argv ) 
{
    int hwidth, dims[3];
    double ***data;
    int cdims[]={0,0,0};
    int periods[]={0,0,0};
    
    MPI_Comm cart_comm;
    ADCL_Vector vec;
    ADCL_Topology topo;
    ADCL_Request request;

    MPI_Init ( &argc, &argv );
    ADCL_Init ();
    
    MPI_Dims_create ( size, 3, cdims );
    MPI_Cart_create ( MPI_COMM_WORLD, 3, cdims, periods, 0, &cart_comm);
    ADCL_Topology_create ( cart_comm, &topo );

    /******************************************************************/
    /* Test 1: hwidth=1, nc=0 */
    hwidth=1;
    dims[0] = DIM0 + 2*hwidth;
    dims[1] = DIM1 + 2*hwidth;
    dims[2] = DIM2 + 2*hwidth;
    ADCL_Vector_allocate ( 3,  dims, 0, hwidth, MPI_DOUBLE, &data, &vec );
    ADCL_Request_create ( vec, topo, ADCL_FNCTSET_NEIGHBORHOOD, &request );
    
    /* start the computation and communication. call the following function
       an arbitrary number of times */ 
    ADCL_Request_start ( request );


    /* if computation/communication is done, free all handles */
    ADCL_Request_free ( &request );
    ADCL_Vector_free ( &vec );
    ADCL_Topology_free ( &topo );
    MPI_Comm_free ( &cart_comm );
    
    ADCL_Finalize ();
    MPI_Finalize ();
    return 0;
}
\end{verbatim}
