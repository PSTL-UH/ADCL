\section{Timer object}

Users might desire to tune a particular communication operation based on a
large codesequence, e.g. an entire iteration of their iterative algorithm or
similar, well defined sections in the code. There are three specific instance
where it makes sense to tune an operation based on a larger code sequence to
potentially includes also computational operations:

\begin{enumerate}

\item for very short running operations, where clock resolution and
  synchronization of processes might be an issue.

\item for non-blocking operations, in which (theoratically) the bulk of the
  communication operation would happen in the background, and thus timing the
  actual function call is nearly irrelevant.

\item for code sections which contain multiple operations that shall be tuned
  simultaniously. Tuning each operation separatly can lead in such a scenario
  to optimal execution time of each operation individually, but not globally
  from the application perspective.

\end{enumerate}

The timer object allows to register an ADCL\_Request with a timer, and add
function calls to start and stop the timer outside of the
ADCL\_Request\_start / \_init() functions.

\begin{verbatim}
int ADCL_Timer_create ( int nreq, ADCL_Request *reqs, ADCL_Timer *timer );
\end{verbatim}
with
\begin{itemize}
\item {\tt nreqs}(IN): number of requests to be associated with the timer object.
\item {\tt reqs}(IN): array of dimension {\tt nreqs} containing the array of
  {\tt ADCL\_Requests} to be associated with the timer object.
\item {\tt timer}(OUT): handle to ADCL timer object.
\end{itemize}


\begin{verbatim}
int ADCL_Timer_free ( ADCL_Timer *timer );
\end{verbatim}
with
\begin{itemize}
\item {\tt timer}(INOUT): handle to the timer object allocated with {\tt
  ADCL\_Timer\-\_create}.  Upon successful completion, the handle will be set
  to {\tt ADCL\_TIMER\_NULL}.
\end{itemize}


\begin{verbatim}
int ADCL_Timer_start ( ADCL_Timer timer );
int ADCL_Timer_stop ( ADCL_Timer timer );
\end{verbatim}
start and stop the timer. The execution time between the start and the stop
function will be stored with each ADCL Request that has been associated with
the provided timer object.


The following shows an example on how to associate a request with a timer
object and how to utilize it in the code.

\begin{verbatim}
int main ( int argc, char ** argv ) 
/**********************************************************************/
{
    /* General variables */
    int hwidth=1, rank, size, it;
    
    /* Definition of the 2-D vector */
    int dims[2], neighbors[4];
    double **data1;
    ADCL_Vmap vmap;
    ADCL_Vector vec1;

    /* Variables for the process topology information */
    int cdims[]={0,0};
    int periods[]={0,0};
    MPI_Comm cart_comm;
    ADCL_Topology topo;
    ADCL_Request req1;
    ADCL_Request *reqs;  
    ADCL_Timer timer;

    /* Initiate the MPI environment */
    MPI_Init ( &argc, &argv );
    MPI_Comm_rank ( MPI_COMM_WORLD, &rank );
    MPI_Comm_size ( MPI_COMM_WORLD, &size );

    /* Describe the neighborhood relations */
    MPI_Dims_create ( size, 2, cdims );
    MPI_Cart_create ( MPI_COMM_WORLD, 2, cdims, periods, 0, &cart_comm);

    /* Initiate the ADCL library and register a topology object with ADCL */
    ADCL_Init ();
    ADCL_Topology_create ( cart_comm, &topo );

    /* allocate vmap and vector object and create a request */
    ADCL_Vmap_halo_allocate ( hwidth, &vmap );
    ADCL_Vector_allocate_generic ( 2,  dims, 0, vmap, MPI_DOUBLE, &data1, &vec1 );
    ADCL_Request_create ( vec1, topo, ADCL_FNCTSET_NEIGHBORHOOD, &req1 );

    /* define timer object */
    ADCL_Timer_create ( 1, &req1, &timer );

    for (it=0; it<MAXIT; it++){
        ADCL_Timer_start( timer );

        /* perform some computation */ 
        ...       
        /* Start the communication */
        ADCL_Request_start ( req1 );

        /* perform some more computation */
        ....
        ADCL_Timer_stop( timer );
    }

    ADCL_Timer_free   ( &timer );
    ADCL_Request_free ( &req1 );
    ADCL_Vector_free  ( &vec1 );
    ADCL_Vmap_free    ( &vmap );
    ADCL_Topology_free ( &topo );
    MPI_Comm_free ( &cart_comm );
    
    ADCL_Finalize ();
    MPI_Finalize ();
    return 0;
}

\end{verbatim}
